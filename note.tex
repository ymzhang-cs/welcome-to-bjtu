\documentclass[a4paper]{ctexart}

\usepackage{multirow}
\usepackage{array,tabularx}
%% longtable包 用于生成长表格
\usepackage{longtable}
\usepackage{booktabs}

%opening
\title{2024年BJTU入学笔记}
\author{张原鸣 \\ 计算机科学与技术专业,2023级 \\ \texttt{ymzhang23@bjtu.edu.cn}}

\begin{document}

\maketitle

\section{前言}

首先,欢迎你来到北京交通大学!

这份笔记是我在2023年入学时记录下来的,并在今年结合了2024年《本科新生必读》的内容进行了修订,希望能够在你刚进入这个陌生环境时带来帮助。

你可以将这篇文章当作一个只保留了(部分)关键信息的新生手册,或是一份临走时的物品清单。事实上,还有很多有用的东西没有包含在内。这需要你自己探索录取通知书内的那本《本科新生必读》。如果下面的内容与官方手册的内容有冲突,那么请你以官方手册为准。

本文档中使用楷体的文字为校内外服务平台名称,第一次出现时会有脚注指明网址。附录对这些网址、一些常用网站和电话进行了整理,方便单独查找。

本文档采用 \LaTeX 排版,代码开源于GitHub\footnote{GitHub Repository: ymzhang-cs/welcome-to-bjtu},并采用 CC BY-NC-SA 4.0 协议共享。如果有人希望与我一起贡献这份笔记,非常欢迎你联系我,或直接在GitHub上提交Pull Request。但如果你只是对这个全新的环境不了解,那么去询问你的班助或许是个更好的选择。

本页脚注附上2024年《本科新生必读》下载链接\footnote{2024年《本科新生必读》 https://zsw.bjtu.edu.cn/data/upload/ueditor/20240708/668bbceb36b89.pdf}。

那么,开始探索这座学校吧!

\newpage

\section{报到手续}

\subsection{需准备资料}
\marginpar{\footnotesize 学生档案自带或告知高中大学地址并邮寄}
\begin{itemize}
	\item 个人档案(学生档案、团员档案)
	\item 录取通知书
	\item 照片:8张以上近期一寸免冠半身彩色照片
	\item 身份证及其扫描复印件:8张以上,正反面于一张A4纸上
\marginpar{\footnotesize 有关户口迁移,参照《本科新生必读》第3页}
	\item 户口迁移证或户口本复印件(首页、户主页、本人页、本人页的变更页)
\end{itemize}


	报到所需材料见第\pageref{携带证件资料} 页 \ref{携带证件资料} 部分。

\subsection{党团组织关系转接}

参照《本科新生必读》第5页。

原先使用\textit{智慧团建}\footnote{智慧团建 https://zhtj.youth.cn/zhtj/}系统的,在开学后进行在线转接。

Tip:忘记密码的提前联系高中团支书重置。

\subsection{辅导员信息}

参考手册信息。学校录取完毕后,各学院辅导员、班主任或班主任助理(由高年级在校生担任)将“一对一、点对点”通知新生,组建班级群,请新生注意接收信息。具体报到流程和要求,请新生及时关注班级群通知。

\subsection{网上预报道}

\begin{description}
	\item[时间] 8月5日
\end{description}

登录\textit{北京交通大学迎新网(下称迎新网)}\footnote{迎新网 https://welcome.bjtu.edu.cn/}。阅读“网上报到须知”,完成“新生必做”环节。

\subsection{完成体育锻炼调查问卷}

访问\textit{迎新网},查看《北京交通大学2023级本科生“入校即测”工作方案》要求,并完成《北京交通大学2023及新生体育锻炼意向调查问卷》,安排假期体育锻炼。

\subsection{床上用品与空调租赁}

\subsubsection{床上用品购买}

\begin{description}
	\item[价格] 480元/套
	\item[时间] 8月12日前
\end{description}

登录\textit{迎新网}进行预订。费用于开学后统一收取。

\subsubsection{空调租赁}

\begin{description}
	\item[价格] 梯度价格详见手册
	\item[时间] 8月12日前
\end{description}

登录\textit{迎新网}进行预订,开学后于新生住宿服务站办理租赁手续。

\subsection{校医院建档签约}

\begin{description}
	\item[时间] 8月15日前
\end{description}

登录\textit{迎新网},查收《校医院给新生的一封信》,并自行完成家医签约和健康档案建档。

\subsection{(未成年)疫苗注射委托书}

截止到2024年8月19日未满18周岁的学生,需家长参照模板填写委托书,模板详见迎新网。

\subsection{新生入学缴费}

\begin{description}
	\item[时间] 8月28日--9月20日
\end{description}

\marginpar{\footnotesize 缴费两周后可登录MIS系统查询、下载电子票据。}

登录\textit{北京交通大学校园信息门户(管理信息系统,MIS)}\footnote{北京交通大学校园信息门户 https://mis.bjtu.edu.cn/},85.财务缴费系统,勾选学费和住宿费,进行支付。

\section{报到}

\subsection{时间}
8月18日 8:00--18:00

\subsection{携带证件资料}
\label{携带证件资料}

\marginpar{\footnotesize 取得学籍后登录\textit{学信网}进行入学资格确认。}

录取通知书、身份证、户口迁移证/户口本复印件(包含首页、户主页、本人页、本人页的变更页)

\section{报到后}
\subsection{新生入学体检与疫苗注射}

\subsubsection{新生入学体检}
入校后进行,体检费用119.5元。包含以下项目:常规检查、血常规、血生化(ALT、Cr)、数字化胸片。

\subsubsection{外省进京疫苗注射}

\begin{description}
	\item[必须接种] 麻风腮疫苗(免费)
	\item[自愿自费接种] 甲肝疫苗(183元/针),水痘疫苗(193元/针),乙肝疫苗(143元/针)
\end{description}

\subsection{英语分级考试}

\begin{description}
	\item[时间] 9月1日 19:00--21:00
\end{description}

\subsection{借记卡激活}

对于未满18岁的本科新生,需由监护人带户口本、出生证明或其他证明直系亲属关系证件、父母及学生的身份证件等至中国银行交大支行代办开卡,具体所需证件请务必提前与中国银行北京交大支行联系,电话:010-62257133。若监护人无法陪同至北京,也可在新生当地开立借记卡,开卡及借记卡使用事宜详情咨询当地中国银行。

\subsubsection{开学时正常领卡}

\begin{description}
	\item[时间] 8月31日至9月13日,具体待通知
\end{description}

开学时发放,中行人员到校办理激活。激活时携带身份证件与联名卡。

\textit{开户行名称:中国银行北京交大支行}

\textit{位于:北京交通大学西门向北 50 米,交大科技大厦一层中国银行}
	
\subsubsection{开学时未领取到卡}

持本人身份证件前往中国银行北京交大支行办理并激活,并将新卡号手动填写至学校财务系统(MIS -> 20.财务系统 -> 个人卡号管理)。

\subsection{职慧V课堂}

9月初可登录,用户名为学号,初始密码为身份证号,登录后可改密码。

职慧V课堂:http://job.bjtu.edu.cn/

\subsection{体育测试}

测试内容:50米、立定跳远、引体向上(男)/1分钟仰卧起坐(女)、1000米(男)/800米(女)。

开学后第一教学周,于课上完成测试内容。

\newpage

\appendix

\section{校园信息服务}

% \begin{tabularx}{\textwidth}{X|X}
\begin{longtable}{m{0.2\textwidth}m{0.8\textwidth}}
	\hline
	\textbf{服务名称} & \textbf{说明} \\
	\hline
	校园网 & \textbf{用途}:使用网络,访问校内资源。\newline\textbf{访问方法}:无线网wifi名称为web.wlan.bjtu, phone.wlan.bjtu。校园网连接后使用上网账号认证,账号为学号,密码为初始密码(登录\textit{迎新网}查看)。\newline\textbf{收费}:8、9月份可免费使用网络,之后在学校微信企业号内进行缴费。\newline\textbf{上网管理}:访问\textit{宽带计费自服务系统}\footnote{宽带计费自服务系统 https://service.bjtu.edu.cn/}可进行密码修改、余额查询、停机与复通等服务。 \\
	\hline
	管理信息系统(校园信息门户,MIS) & \textbf{用途}:学校的综合信息门户,使用应用资源并办理校内业务。\newline\textbf{登录方法}:用户名为学号,初始密码见\textit{迎新网}。 \\
	\hline
	微信信息门户(微信企业号) & \textbf{用途}:学习各部门通知发布,还包含四六级考试信息、交网费、成绩查询等业务。\newline\textbf{关注方法}:详见手册。验证信息为\textit{MIS系统}密码。 \\
	\hline
	电子邮件系统 & \textbf{用途}:使用带有学校后缀的个人邮箱。\newline\textbf{使用方法}:访问\textit{管理信息系统}或\textit{电子邮箱系统}\footnote{电子邮箱系统 https://mail.bjtu.edu.cn/},或通过第三方邮箱客户端访问。允许设置一次别名。 \\
	\hline
	校园一卡通 & \textbf{用途}:校内消费、身份认证等。\newline\textbf{充值方式}:1.在指定食堂现金充值(学一餐厅、学活餐厅、东校区餐厅、学苑餐厅)。2.在一卡通自助终端使用绑定的中国银行银行卡进行圈存转账。3.使用\textbf{完美校园APP或完美校园微信小程序}充值,支持使用微信及绑定的中国银行银行卡。\newline\textbf{初始密码}:初始密码为身份证后六位(X用0代替)。 \\
	\hline
	
% \end{tabularx}
\end{longtable}

\section{校内外服务平台指北}

\begin{tabularx}{\textwidth}{XX}
	\toprule
	\textbf{平台名称} & \textbf{网址} \\
	\midrule
	北京交通大学迎新网 & https://welcome.bjtu.edu.cn/ \\
	北京交通大学本科生院 & https://bksy.bjtu.edu.cn/ \\
	北京交通大学校园信息门户\newline(管理信息系统,MIS) & https://mis.bjtu.edu.cn/ \\
	北京交通大学电子邮箱系统 & https://mail.bjtu.edu.cn/ \\
	北京交通大学宽带计费自服务系统 & https://service.bjtu.edu.cn/ \\
	北京交通大学 VPN & http://highpc.bjtu.edu.cn/vpn/ \\
	中国高等教育学生信息网 & https://www.chsi.com.cn/ \\
	职慧V课堂 & http://job.bjtu.edu.cn/ \\
	中国大学 MOOC & https://www.icourse163.org/ \\
	知行论坛(仅限教育邮箱注册) & https://zhixing.bjtu.edu.cn/ \\

	\bottomrule
\end{tabularx}

\section{校内外电话黄页}

\begin{tabularx}{\textwidth}{XX}
	\toprule
	\textbf{名称} & \textbf{电话} \\
	\midrule
	保卫处报警电话 & 010-51687110 \\
	校医院急诊值班室 & 010-51683647 \\
	学生公寓管理中心 & 010-51685963 \\
	信息中心 & 010-51688446 \newline 010-51688476 \\
	中国银行北京交大支行 & 010-62257133 \\
	中国银行 & 95566 \\
	\bottomrule
\end{tabularx}

\end{document}
